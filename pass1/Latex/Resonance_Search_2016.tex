\documentclass[letterpaper,12pt]{article}
%documentclass[superscriptaddress,preprintnumbers,amsmath,amssymb,aps,11pt]{revtex4}
%\usepackage[]{authblk}
%\usepackage{graphics}
\usepackage[dvipdf]{graphics}
%\usepackage{subfig}  % For subfloats
\usepackage{color}
\usepackage[usenames,dvipsnames]{xcolor}
\usepackage{epsfig}
\usepackage{wrapfig}
\usepackage{rotating}
\usepackage{caption}
%\usepackage{subcaption}
\usepackage{subfig}
\usepackage{authblk}
\usepackage{url}

\oddsidemargin = -14mm
\topmargin = -2.9cm
\textwidth = 19cm
\textheight = 24cm

\def \rarr {\rightarrow}
\def \grinp {\includegraphics}
\def \tw {\textwidth}
\def\dfrac#1#2{\displaystyle{{#1}\over{#2}}}
\def \dstl {\displaystyle}
\definecolor{GREEN}{rgb}{0.,0.8,0}
\definecolor{RED}{rgb}{1,0,0}
\definecolor{ORANGE}{rgb}{1,0.5,0}

\author{Bump hunt folks}
\title{Resonance search analysis of 2016 HPS spring run data.}

\begin{document}

\maketitle

\tableofcontents
\newpage

\section*{Introduction}

The Heavy Photon Search (HPS) experiment has capability to search for a so called heavy photon ($A^{\prime}$) with two complementary methods.


\section{Data set}
Describe the data, beam energy, beam current, target runs, etc.

\section{Event Selections}
Describe list of cuts for the final event selection, and justification of each cut

Main goal of event selections cuts is to maximize signal sensitivity.

\section{Bump hunt analysis}


\section{Study of systematics}
Here goes studies on systematics

 
\end{document}
