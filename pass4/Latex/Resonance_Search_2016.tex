\documentclass[letterpaper,12pt]{article}
%documentclass[superscriptaddress,preprintnumbers,amsmath,amssymb,aps,11pt]{revtex4}
%\usepackage[]{authblk}
%\usepackage{graphics}
\usepackage[dvipdf]{graphics}
%\usepackage{subfig}  % For subfloats
\usepackage{color}
\usepackage[usenames,dvipsnames]{xcolor}
\usepackage{epsfig}
\usepackage{wrapfig}
\usepackage{rotating}
\usepackage{caption}
%\usepackage{subcaption}
\usepackage{subfig}
\usepackage{authblk}
\usepackage{url}

\oddsidemargin = -14mm
\topmargin = -2.9cm
\textwidth = 19cm
\textheight = 24cm

\def \rarr {\rightarrow}
\def \grinp {\includegraphics}
\def \tw {\textwidth}
\def\dfrac#1#2{\displaystyle{{#1}\over{#2}}}
\def \dstl {\displaystyle}
\definecolor{GREEN}{rgb}{0.,0.8,0}
\definecolor{RED}{rgb}{1,0,0}
\definecolor{ORANGE}{rgb}{1,0.5,0}

\author{Bump hunt folks}
\title{Resonance search analysis of 2016 HPS spring run data.}

\begin{document}

\maketitle

\tableofcontents
\newpage

\section*{Introduction}

The Heavy Photon Search (HPS) experiment has capability to search for a so called heavy photon ($A^{\prime}$) with two complementary methods.


\section{Data set}
Describe the data, beam energy, beam current, target runs, etc.

\section{Event Selections}
This section describes all the cuts that are applied to get the final vertex candidate distribution. 
The main goal of event selection cuts is to maximize signal sensitivity.

In this analysis only events that with ``Pair1'' trigger are used.

\subsection{Cluster timing cuts}

The readout window of ECal FADC data is 200 ns. Clusters coming from the physics events, that made generated the trigger, are located in a narrow time range in the readout window around $\mathrm{t = 56\; ns}$.


\subsection{Track $\chi^{2}$ cuts}
\subsection{Track-Cluster Matching}
\subsection{WAB Suppresion cuts}
Describe L1 and $\mathrm{d_{0}}$ cuts.

\section{Parametrization of Mass resolution.}

\section{Bump hunt analysis}


\section{Study of systematics}
Here goes studies on systematics

 
\end{document}
