\documentclass[letterpaper,12pt]{article}
%documentclass[superscriptaddress,preprintnumbers,amsmath,amssymb,aps,11pt]{revtex4}
%\usepackage[]{authblk}
%\usepackage{graphics}
\usepackage[dvipdf]{graphics}
%\usepackage{subfig}  % For subfloats
\usepackage{color}
\usepackage[usenames,dvipsnames]{xcolor}
\usepackage{epsfig}
\usepackage{wrapfig}
\usepackage{rotating}
\usepackage{caption}
%\usepackage{subcaption}
\usepackage{subfig}
\usepackage{authblk}
\usepackage{url}

\oddsidemargin = -14mm
\topmargin = -2.9cm
\textwidth = 19cm
\textheight = 24cm

\def \rarr {\rightarrow}
\def \grinp {\includegraphics}
\def \tw {\textwidth}
\def\dfrac#1#2{\displaystyle{{#1}\over{#2}}}
\def \dstl {\displaystyle}
\definecolor{GREEN}{rgb}{0.,0.8,0}
\definecolor{RED}{rgb}{1,0,0}
\definecolor{ORANGE}{rgb}{1,0.5,0}

\author{Bump hunt folks}
\title{Resonance search analysis of 2016 HPS spring run data.}

\begin{document}

\maketitle

\tableofcontents
\newpage

\section*{Introduction}

The Heavy Photon Search (HPS) experiment has capability to search for a so called heavy photon ($A^{\prime}$) with two complementary methods.


\section{Data set}
Describe the data, beam energy, beam current, target runs, etc.

\section{Event Selections}
This section describes all the cuts that are applied to get the final vertex candidate distribution. 
The main goal of event selection cuts is to maximize signal sensitivity.

In this analysis only events with ``Pair1'' trigger (see \cite{TriggerNote} for the description of HPS triggers) are used.

\subsection{Cluster timing cuts}
The readout window of ECal FADC data is 200 ns. Clusters coming from the physics events, that generated the trigger, are located in a narrow time range (few ns width because of the trigger jitter) in the readout window around $\mathrm{t = 56\; ns}$.
%%%%%%%%%%%%%%%%%%%%%%%%%% F I G U R E %%%%%%%%%%%%%%%%%%%%%%%%%%%%%%%
\begin{figure}[!htb]
 \centering
 \subfloat{\label{fig:clE_t:Top}\grinp[width=0.45\tw]{Figs/cl_t_E_Top.pdf}}
 \subfloat{\label{fig:clE_t:Bot}\grinp[width=0.45\tw]{Figs/cl_t_E_Bot.pdf}}
 \caption{``time vs Energy'' distributions of ECal clusters in the Top (Left) and Bottom (Right) half. Red curves in the in the right plot indicate cuts that are applied to clusters in the bottom half for the initial cluster selection. See the text for the description of the difference between left and right plots.}
 \label{fig:cl_E_T_plots}
\end{figure}
%%%%%%%%%%%%%%%%%%%%%%%%%% F I G U R E %%%%%%%%%%%%%%%%%%%%%%%%%%%%%%%
In Fig.\ref{fig:cl_E_T_plots} shown ``time vs Energy'' distributions of ECal clusters in the Top (Left) and Bottom (Right) halves. The bulge of events in the right plot are clusters that generated the trigger, and also trigger time is defined by these clusters. The noticeable energy dependence is due to the so called ``time walk Corrections'' \cite{ECalTimeCorr}. During initial event selection only clusters that are inside the outlined red curves are used, since the rest are accidentals that didn't come from the beam bunch generating the trigger. One can notice that for the clusters in the top half, in addition to the central bulge, there is an extra occupancy of events in region ($\mathrm{ 40\;ns < t_{cl} < 70\;ns }$). This is because the coincidence time between clusters in the ``Pair'' trigger was $\mathrm{12\;ns}$ \cite{TriggerNote}, and the trigger time is determined by the bottom cluster. Unlike to clusters in the bottom half, in the initial event selection, we have not cut on time of the top cluster, but rater we have applied cut on the cluster time difference between top and bottom clusters.
\subsubsection{Ad hoc ECal time corrections}
The next step is to cut pairs of top-bottom clusters that are far from each other in terms of time.
%%%%%%%%%%%%%%%%%%%%%%%%%% F I G U R E %%%%%%%%%%%%%%%%%%%%%%%%%%%%%%%
\begin{figure}[!htb]
 \centering
 \grinp[width=0.75\tw]{Figs/cl_dt_UncrAndCorr.pdf}
 \caption{Time difference between top and bottom clusters at high $\mathrm{E_{sum}}$ region. Dashed red lines show uncorrected clusters, and the blue curve show corrected clusters.}
 \label{fig:clTimeUncorCorr}
\end{figure}
%%%%%%%%%%%%%%%%%%%%%%%%%% F I G U R E %%%%%%%%%%%%%%%%%%%%%%%%%%%%%%%
During the analysis is was found that ECal cluster times can be improved, in particular the dashed red histogram in Fig.\ref{fig:clTimeUncorCorr} shows the time difference between top and bottom clusters\footnote{For the sake of better visualization, the plot shows doesn't fully show the entire central peak. } at high $\mathrm{E_{sum}}$ region ($\mathrm{1.9\;GeV  < E_{Sum} < 2.4\; GeV}$). As one can see there is a bump at around $\mathrm{2\;ns}$, while the at $\mathrm{-2\;ns}$ there is no clear bump. This suggests that time offsets of some crystals might be wrong.
%%%%%%%%%%%%%%%%%%%%%%%%%% F I G U R E %%%%%%%%%%%%%%%%%%%%%%%%%%%%%%%
\begin{figure}[!htb] 
 \centering
 \subfloat{\label{fig:timeCorr_clyxc}\grinp[width=0.95\tw]{Figs/ECal_YX_timeCorr.pdf}}\\
 \subfloat{\label{fig:timeCorrGrTop}\grinp[width=0.95\tw]{Figs/gr_TopCorrections.pdf}}\\
 \subfloat{\label{fig:timeCorrGrBot}\grinp[width=0.95\tw]{Figs/gr_BotCorrections.pdf}}\\
 \caption{Time Corrections for each crystal.}
 \label{fig:CrystalTimeCorr}
\end{figure}
%%%%%%%%%%%%%%%%%%%%%%%%%% F I G U R E %%%%%%%%%%%%%%%%%%%%%%%%%%%%%%%
To check this, for each crystal the time difference between that crystal and it's pair (in opposite half) crystal is constructed. The Top 2D plot of
Fig. \ref{fig:CrystalTimeCorr} shows mean values of each of crystals. 
The middle and the bottom plots show same mean values as a function of crystal $\mathrm{X}$ index for Top and Bottom crystal respectively. Different markers show different rows. 
As one can see crystals (-18, -1) and (-6, -3) are shifted from their immediate neighbors  by about 2 ns. There are some other crystals which are shifted significantly too (but less than 2ns). These include for example crystals (-16, -5), (10, -5), (10, 1) and (10, 2). In addition to this, we see that there is a general crystal X index (and slightly Y index) dependence too. In reality crystal X index is correlated to the charged particle energy too, and the original dependence might be not on X but on energy. Studying it is out of the scope of this note, and here for each crystal we have corrected the time, by subtracting these calculated mean values from the reconstructed cluster time.
After the correction the cluster time difference is depicted by blue solid line in Fig.\ref{fig:clTimeUncorCorr}. One can see that the excess of events at 2 ns disappeared. Dips and peaks between bumps indicating difference beam bunches also got sharper, which is an indication of an improvement of the cluster time resolution.

\subsubsection{Fitting Cluster time difference}
After correction of individual cluster times, the Top-Bottom cluster time difference was fitted with a following function:
\begin{equation}
 \displaystyle \mathrm{F = \sum_{i = 0}^{N_{peak}} a_{i}\cdot\left(Gaus(x - \mu^{1}_{i}, \sigma^{1}_{i}) + b\cdot Gaus(x - \mu^{2}_{i}, \sigma^{2}_{i})  \right)}
\end{equation}
where $\mathrm{N_{peak}}$ is the number of peaks. Each peak is described by the sum of two Gaussian functions $\mathrm{Gaus(x - \mu^{1}_{i}, \sigma^{1}_{i})}$ and $\mathrm{Gaus(x - \mu^{2}_{i}, \sigma^{2}_{i})}$ with their amplitude ratio ``b'.
The parameter ''b`` is the same for all peaks. 
In the fit, free parameters are $\mathrm{a_{i},\; \mu^{1}_{i},\; \sigma^{1}_{i},\; \mu^{2}_{i},\; \sigma^{2}_{i},\; b }$. 

The fit result is shown in Fig.\ref{fig:cldTFits}. Different peak components of the
%%%%%%%%%%%%%%%%%%%%%%%%%% F I G U R E %%%%%%%%%%%%%%%%%%%%%%%%%%%%%%%
\begin{figure}[!htb]
 \centering
 \subfloat[]{\label{fig:cldTFitLin1}\grinp[width=0.32\tw]{Figs/Calo_dt_Fit_lin1.pdf}}
 \subfloat[]{\label{fig:cldTFitLog1}\grinp[width=0.32\tw]{Figs/Calo_dt_Fit_log1.pdf}}
 \subfloat[]{\label{fig:cldTFitZoom1}\grinp[width=0.32\tw]{Figs/Calo_dt_Fit_zoom1.pdf}}
 \caption{Fit of the Top and Bottom cluster time difference. Left: linear scale, Middle: Log scale and right: Linear scale but includes only low magnitude peaks. Main Gaussian functions are represented by solid lines, and the secondary Gaussian (with wider width and lower magnitude) are represented by dashed lines.}
 \label{fig:cldTFits}
\end{figure}
%%%%%%%%%%%%%%%%%%%%%%%%%% F I G U R E %%%%%%%%%%%%%%%%%%%%%%%%%%%%%%%
function are depicted by different color. The main Gaussian function of each peak is represented by a solid line, while the secondary Gaussian is represented with a dashed line. One can see that this function fits the distribution reasonably well.

Then in order to determine the optimal cut on the cluster time difference, we will use the value, which maximizes the $\mathrm{\dfrac{S}{\displaystyle \sqrt{S + Bgr}}}$ ratio, where ''S`` is the signal (in our case the central peak), and ''S + Bgr`` is the signal plus Background (the total fit function).
%%%%%%%%%%%%%%%%%%%%%%%%%% F I G U R E %%%%%%%%%%%%%%%%%%%%%%%%%%%%%%%
\begin{figure}[!htb]
 \centering
 \grinp[width=0.75\tw]{Figs/dt_cur_optimize1_zoom.pdf}
 \caption{The $\mathrm{\frac{\displaystyle S}{\displaystyle \sqrt{S + Bgr}}} $ ratio as a function of cluster time difference cut. The Dashed line indicates the maximum of the function.}
 \label{fig:cl_dtOptimumCut}
\end{figure}
%%%%%%%%%%%%%%%%%%%%%%%%%% F I G U R E %%%%%%%%%%%%%%%%%%%%%%%%%%%%%%%
The $\mathrm{\frac{\displaystyle S}{\displaystyle \sqrt{S + Bgr}}} $ ratio as a function of cluster time difference cut is shown in Fig.\ref{fig:cl_dtOptimumCut}, where the maximum value at $\mathrm{\Delta t < 1.43\;ns}$ is indicated by a vertical dashed line.


\subsection{Track $\chi^{2}$ cuts}
\subsection{Track-Cluster Matching}
This section should include both spatial and time matching of clusters and tracks.
\subsection{WAB Suppresion cuts}
Describe L1 and $\mathrm{d_{0}}$ cuts. \\
Mention why L1 is important, addition of suppresing WABs, it also significantly
improves the mass resolution.


\section{Parametrization of Mass resolution.}

\section{Bump hunt analysis}


\section{Study of systematics}
Here goes studies on systematics

\begin{thebibliography}{55}
 \bibitem{TriggerNote} Kyle McCarty, Valery Kubarovsky and Benjamin Raydo, ``Description and Tuning of the HPS Trigger'', HPS Note 2018-002
 \bibitem{ECalTimeCorr} Holly Szumila-Vance, ``HPS Ecal Timing Calibration for the
Spring 2015 Engineering Run'', HPS Note 2015-011.
\end{thebibliography}

 
\end{document}
